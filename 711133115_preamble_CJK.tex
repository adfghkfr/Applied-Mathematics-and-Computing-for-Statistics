\documentclass[12pt, a4paper]{article} 
\usepackage{fontspec} % Font selection for XeLaTeX; see fontspec.pdf. 

\usepackage{xeCJK}	% 中文使用 XeCJK,利用 \setCJKmainfont 定義中文內文、粗體與斜體的字型
%\usepackage[BoldFont, SlantFont]{xeCJK}
% 中文使用 XeCJK,並模擬粗體與斜體(\textbf{ } \textit{ })

\defaultfontfeatures{Mapping=tex-text} 
% to support TeX conventions like ``---''
\usepackage{xunicode} 
% Unicode support for LaTeX character names(accents, European chars, etc)
\usepackage{xltxtra} 				
% Extra customizations for XeLaTeX
\usepackage{amsmath, amssymb}
\usepackage{enumerate}
\usepackage{graphicx, subfig, float, wrapfig} 
% support the \includegraphics command and options
\usepackage[outercaption]{sidecap} 
%[options]=[outercaption], [innercaption], [leftcaption], [rightcaption]
\usepackage{array, booktabs}
\usepackage{color, xcolor}
% 跨頁的超長表格;lscape是旋轉此類表格的
\usepackage{longtable, lscape}              
% 巨集,使表格加註解更容易(手冊p169)
\usepackage{threeparttable}      
% 讓表格編起來更美的套件(手冊p166),編輯跨列標題重覆的表格(手冊p182)           
\usepackage{multirow, booktabs}             
\usepackage{colortbl}     
% 直接將 latex 碼轉換成顯示文字   
\usepackage{listings}						
% 新段落前加一空行,不使用縮排

\usepackage[parfill]{parskip} 				
% 新段落前加一空行,不使用縮排

\usepackage{geometry} 
% See geometry.pdf to learn the layout options. There are lots.
%\usepackage[left=3in,right=3in,top=2in,bottom=2in]{geometry} 

\usepackage{url}

%.....表格標題註解之巨集套件.....%
% for Reference
\usepackage{natbib}			
% for Indexing				
\usepackage{makeidx}		
% Activate to begin paragraphs with an empty line rather than an indent

%章節設定
\usepackage{titlesec, titletoc,CJKnumb}		


%-----------------------------------------------------------------
%  中英文內文字型設定
%\setCJKmainfont							% 設定中文內文字型
%	[
		%BoldFont=Microsoft YaHei	    %定義粗體的字型(Win)
%		BoldFont=蘋果儷中黑	    		%定義粗體的字型(Mac)
%	]
%	{新細明體}						% 設定中文內文字型(Win)
%	{宋體-繁}							% 設定中文內文字型(Mac)	
%\setmainfont{Times New Roman}		% 設定英文內文字型
%\setsansfont{Arial}					% 無襯字字型 used with {\sffamily ...}
%\setsansfont[Scale=MatchLowercase,Mapping=tex-text]{Gill Sans}
%\setmonofont{Courier New}			% 等寬字型 used with {\ttfamily ...}
%\setmonofont[Scale=MatchLowercase]{Andale Mono}


% 英文字型
%\newfontfamily{\E}{Calibri}				
%\newfontfamily{\A}{Arial}
%\newfontfamily{\C}[Scale=0.9]{Arial}
%\newfontfamily{\R}{Times New Roman}
%\newfontfamily{\TT}[Scale=0.8]{Times New Roman}

% 中文字型
%\newCJKfontfamily{\MB}{微軟正黑體}				% 等寬及無襯線字體 Win
%\newCJKfontfamily{\SM}[Scale=0.8]{新細明體}	% 縮小版(Win)
%\newCJKfontfamily{\K}{標楷體}                	% Windows下的標楷體
%\newCJKfontfamily{\BB}{Microsoft YaHei}		% 粗體 Win
%\newCJKfontfamily{\CF}{cwTeX Q Fangsong Medium}	% CwTex 仿宋體
%\newCJKfontfamily{\CB}{cwTeX Q Hei Bold}			% CwTex 粗黑體
%\newCJKfontfamily{\CK}{cwTeX Q Kai Medium}   	% CwTex 楷體
%\newCJKfontfamily{\CM}{cwTeX Q Ming Medium}		% CwTex 明體
%\newCJKfontfamily{\CR}{cwTeX Q Yuan Medium}		% CwTex 圓體
%----------------------------------------------------------
%  中英文內文字型設定
\setCJKmainfont							% 設定中文內文字型
	[
%		BoldFont=Microsoft YaHei	    % 定義粗體的字型(Win)
		BoldFont=黑體-繁	    		    % 定義粗體的字型(Mac)
	]
%	{新細明體}						    % 設定中文內文字型(Win)
	{宋體-繁}							% 設定中文內文字型(Mac)	
\setmainfont{Times New Roman}		    % 設定英文內文字型
\setsansfont{Arial}					    % 無襯字字型 used with {\sffamily ...}
%\setsansfont[Scale=MatchLowercase,Mapping=tex-text]{Gill Sans}
\setmonofont{Courier New}			    % 等寬字型 used with {\ttfamily ...}
%\setmonofont[Scale=MatchLowercase]{Andale Mono}
% 其他字型(隨使用的電腦安裝的字型不同,用註解的方式調整(打開或關閉))
% 英文字型
\newfontfamily{\A}{Arial}
\newfontfamily{\C}{Cochin}				
\newfontfamily{\SA}[Scale=0.9]{Arial}
\newfontfamily{\R}{Times New Roman}
\newfontfamily{\ST}[Scale=0.8]{Times New Roman}
% 中文字型
%\newCJKfontfamily{\MB}{微軟正黑體}				    % 等寬及無襯線字體 Win
\newCJKfontfamily{\MB}{黑體-繁}				        % 等寬及無襯線字體 Mac
%\newCJKfontfamily{\SM}[Scale=0.8]{新細明體}	        % 縮小版(Win)
\newCJKfontfamily{\SM}[Scale=0.8]{宋體-繁}	        % 縮小版(Mac)
%\newCJKfontfamily{\K}{標楷體}                	    % Windows下的標楷體
\newCJKfontfamily{\K}{黑體-繁}               	    % Mac下的標楷體
%\newCJKfontfamily{\BB}{Microsoft YaHei}		    % 粗體 Win
\newCJKfontfamily{\BB}{宋體-繁}		                % 粗體 Mac
% 以下為自行安裝的字型:CwTex 組合
%\newCJKfontfamily{\CF}{cwTeX Q Fangsong Medium}	% CwTex 仿宋體
%\newCJKfontfamily{\CB}{cwTeX Q Hei Bold}			% CwTex 粗黑體
%\newCJKfontfamily{\CK}{cwTeX Q Kai Medium}   	    % CwTex 楷體
%\newCJKfontfamily{\CM}{cwTeX Q Ming Medium}		% CwTex 明體
%\newCJKfontfamily{\CR}{cwTeX Q Yuan Medium}		% CwTex 圓體
%-------------------------------------------------------------
\XeTeXlinebreaklocale "zh"             		
%這兩行一定要加,中文才能自動換行
\XeTeXlinebreakskip = 0pt plus 1pt     		
%這兩行一定要加,中文才能自動換行
%-----------------------------------------------------------------------------------------------------------------------


\newcommand{\cw}{\texttt{cw}\kern-.6pt\TeX}	% 這是 cwTex 的 logo 文字
\newcommand{\imgdir}{/Users/guoyixuan/Documents/NTPU_STAT/images/}				% 設定圖檔的目錄位置
\renewcommand{\tablename}{表}					% 改變表格標號文字為中文的「表」(預設為 Table)
\renewcommand{\figurename}{圖}				% 改變圖片標號文字為中文的「圖」(預設為 Figure)
\usepackage{booktabs}


% 設定顏色 see color Table: http://latexcolor.com
\usepackage{color}
\definecolor{slight}{gray}{0.9}				
\definecolor{airforceblue}{rgb}{0.36, 0.54, 0.66} 
\definecolor{arylideyellow}{rgb}{0.91, 0.84, 0.42}
\definecolor{babyblue}{rgb}{0.54, 0.81, 0.94}
\definecolor{cadmiumred}{rgb}{0.89, 0.0, 0.13}
\definecolor{coolblack}{rgb}{0.0, 0.18, 0.39}
\definecolor{beaublue}{rgb}{0.74, 0.83, 0.9}
\definecolor{beige}{rgb}{0.96, 0.96, 0.86}
\definecolor{bisque}{rgb}{1.0, 0.89, 0.77}
\definecolor{gray(x11gray)}{rgb}{0.75, 0.75, 0.75}
\definecolor{limegreen}{rgb}{0.2, 0.8, 0.2}
\definecolor{splashedwhite}{rgb}{1.0, 0.99, 1.0}

%---------------------------------------------------------------------
% 映出程式碼 \begin{lstlisting} 的內部設定
\lstset
{	language=[LaTeX]TeX,
    breaklines=true,
    %basicstyle=\tt\scriptsize,
    basicstyle=\tt\normalsize,
    keywordstyle=\color{blue},
    identifierstyle=\color{black},
    commentstyle=\color{limegreen}\itshape,
    stringstyle=\rmfamily,
    showstringspaces=false,
    %backgroundcolor=\color{splashedwhite},
    backgroundcolor=\color{slight},
    frame=single,							%default frame=none 
    rulecolor=\color{gray(x11gray)},
    framerule=0.4pt,							%expand outward 
    framesep=3pt,							%expand outward
    xleftmargin=3.4pt,		%to make the frame fits in the text area. 
    xrightmargin=3.4pt,		%to make the frame fits in the text area. 
    tabsize=2				%default :8 only influence the lstlisting and lstinline.
}

% 映出程式碼 \begin{lstlisting} 的內部設定 for Python codes
%\lstset{language=Python}
%\lstset{frame=lines}
%\lstset{basicstyle=\SCP\normalsize}
%\lstset{keywordstyle=\color{blue}}
%\lstset{commentstyle=\color{airforceblue}\itshape}
%\lstset{backgroundcolor=\color{beige}}

%%%數學式定義定理範例設定%%%
%%theroemstyle 需要使用的套件%%
\usepackage{amsthm}						
\theoremstyle{plain}

%%Definition獨立編號%%
\newtheorem{de}{定義}[section]
%%Theorem獨立編號%%
\newtheorem{thm}{{\MB 定理}}[section]		
%%lemma和theorem共同編號%%
\newtheorem{lemma}[thm]{Lemma}	
%%Example獨立編號%%
\newtheorem{ex}{{\A Example}}			

\newtheorem{cor}{Corollary}[section]		%not used here
\newtheorem{exercise}{EXERCISE}			%not used here
\newtheorem{re}{\emph{Result}}[section]	%not used here
\newtheorem{axiom}{AXIOM}				%not used here
\renewcommand{\proofname}{證明}			%not used here
\newcounter{quiz}						% start a simple and new counter
\setcounter{quiz}{1}						% start to count from 1
