\documentclass[12pt, a4paper]{article} 
\usepackage{fontspec} % Font selection for XeLaTeX; see fontspec.pdf. 

\usepackage{xeCJK}	% 中文使用 XeCJK,利用 \setCJKmainfont 定義中文內文、粗體與斜體的字型
%\usepackage[BoldFont, SlantFont]{xeCJK}
% 中文使用 XeCJK,並模擬粗體與斜體(\textbf{ } \textit{ })

\defaultfontfeatures{Mapping=tex-text} 
% to support TeX conventions like ``---''
\usepackage{xunicode} 
% Unicode support for LaTeX character names(accents, European chars, etc)
\usepackage{xltxtra} 				
% Extra customizations for XeLaTeX
\usepackage{amsmath, amssymb}
\usepackage{enumerate}
\usepackage{graphicx, subfig, float, wrapfig} 
% support the \includegraphics command and options
\usepackage[outercaption]{sidecap} 
%[options]=[outercaption], [innercaption], [leftcaption], [rightcaption]
\usepackage{array, booktabs}
\usepackage{color, xcolor}
% 跨頁的超長表格;lscape是旋轉此類表格的
\usepackage{longtable, lscape}              
% 巨集,使表格加註解更容易(手冊p169)
\usepackage{threeparttable}      
% 讓表格編起來更美的套件(手冊p166),編輯跨列標題重覆的表格(手冊p182)           
\usepackage{multirow, booktabs}             
\usepackage{colortbl}     
% 直接將 latex 碼轉換成顯示文字   
\usepackage{listings}						
% 新段落前加一空行,不使用縮排

\usepackage[parfill]{parskip} 				
% 新段落前加一空行,不使用縮排

\usepackage{geometry} 
% See geometry.pdf to learn the layout options. There are lots.
%\usepackage[left=3in,right=3in,top=2in,bottom=2in]{geometry} 

\usepackage{url}

%.....表格標題註解之巨集套件.....%
% for Reference
\usepackage{natbib}			
% for Indexing				
\usepackage{makeidx}		
% Activate to begin paragraphs with an empty line rather than an indent

%章節設定
\usepackage{titlesec, titletoc,CJKnumb}		


%-----------------------------------------------------------------
%  中英文內文字型設定
%\setCJKmainfont							% 設定中文內文字型
%	[
		%BoldFont=Microsoft YaHei	    %定義粗體的字型(Win)
%		BoldFont=蘋果儷中黑	    		%定義粗體的字型(Mac)
%	]
%	{新細明體}						% 設定中文內文字型(Win)
%	{宋體-繁}							% 設定中文內文字型(Mac)	
%\setmainfont{Times New Roman}		% 設定英文內文字型
%\setsansfont{Arial}					% 無襯字字型 used with {\sffamily ...}
%\setsansfont[Scale=MatchLowercase,Mapping=tex-text]{Gill Sans}
%\setmonofont{Courier New}			% 等寬字型 used with {\ttfamily ...}
%\setmonofont[Scale=MatchLowercase]{Andale Mono}


% 英文字型
%\newfontfamily{\E}{Calibri}				
%\newfontfamily{\A}{Arial}
%\newfontfamily{\C}[Scale=0.9]{Arial}
%\newfontfamily{\R}{Times New Roman}
%\newfontfamily{\TT}[Scale=0.8]{Times New Roman}

% 中文字型
%\newCJKfontfamily{\MB}{微軟正黑體}				% 等寬及無襯線字體 Win
%\newCJKfontfamily{\SM}[Scale=0.8]{新細明體}	% 縮小版(Win)
%\newCJKfontfamily{\K}{標楷體}                	% Windows下的標楷體
%\newCJKfontfamily{\BB}{Microsoft YaHei}		% 粗體 Win
%\newCJKfontfamily{\CF}{cwTeX Q Fangsong Medium}	% CwTex 仿宋體
%\newCJKfontfamily{\CB}{cwTeX Q Hei Bold}			% CwTex 粗黑體
%\newCJKfontfamily{\CK}{cwTeX Q Kai Medium}   	% CwTex 楷體
%\newCJKfontfamily{\CM}{cwTeX Q Ming Medium}		% CwTex 明體
%\newCJKfontfamily{\CR}{cwTeX Q Yuan Medium}		% CwTex 圓體
%----------------------------------------------------------
%  中英文內文字型設定
\setCJKmainfont							% 設定中文內文字型
	[
%		BoldFont=Microsoft YaHei	    % 定義粗體的字型(Win)
		BoldFont=黑體-繁	    		    % 定義粗體的字型(Mac)
	]
%	{新細明體}						    % 設定中文內文字型(Win)
	{宋體-繁}							% 設定中文內文字型(Mac)	
\setmainfont{Times New Roman}		    % 設定英文內文字型
\setsansfont{Arial}					    % 無襯字字型 used with {\sffamily ...}
%\setsansfont[Scale=MatchLowercase,Mapping=tex-text]{Gill Sans}
\setmonofont{Courier New}			    % 等寬字型 used with {\ttfamily ...}
%\setmonofont[Scale=MatchLowercase]{Andale Mono}
% 其他字型(隨使用的電腦安裝的字型不同,用註解的方式調整(打開或關閉))
% 英文字型
\newfontfamily{\A}{Arial}
\newfontfamily{\C}{Cochin}				
\newfontfamily{\SA}[Scale=0.9]{Arial}
\newfontfamily{\R}{Times New Roman}
\newfontfamily{\ST}[Scale=0.8]{Times New Roman}
% 中文字型
%\newCJKfontfamily{\MB}{微軟正黑體}				    % 等寬及無襯線字體 Win
\newCJKfontfamily{\MB}{黑體-繁}				        % 等寬及無襯線字體 Mac
%\newCJKfontfamily{\SM}[Scale=0.8]{新細明體}	        % 縮小版(Win)
\newCJKfontfamily{\SM}[Scale=0.8]{宋體-繁}	        % 縮小版(Mac)
%\newCJKfontfamily{\K}{標楷體}                	    % Windows下的標楷體
\newCJKfontfamily{\K}{黑體-繁}               	    % Mac下的標楷體
%\newCJKfontfamily{\BB}{Microsoft YaHei}		    % 粗體 Win
\newCJKfontfamily{\BB}{宋體-繁}		                % 粗體 Mac
% 以下為自行安裝的字型:CwTex 組合
%\newCJKfontfamily{\CF}{cwTeX Q Fangsong Medium}	% CwTex 仿宋體
%\newCJKfontfamily{\CB}{cwTeX Q Hei Bold}			% CwTex 粗黑體
%\newCJKfontfamily{\CK}{cwTeX Q Kai Medium}   	    % CwTex 楷體
%\newCJKfontfamily{\CM}{cwTeX Q Ming Medium}		% CwTex 明體
%\newCJKfontfamily{\CR}{cwTeX Q Yuan Medium}		% CwTex 圓體
%-------------------------------------------------------------
\XeTeXlinebreaklocale "zh"             		
%這兩行一定要加,中文才能自動換行
\XeTeXlinebreakskip = 0pt plus 1pt     		
%這兩行一定要加,中文才能自動換行
%-----------------------------------------------------------------------------------------------------------------------


\newcommand{\cw}{\texttt{cw}\kern-.6pt\TeX}	% 這是 cwTex 的 logo 文字
\newcommand{\imgdir}{/Users/guoyixuan/Documents/NTPU_STAT/images/}				% 設定圖檔的目錄位置
\renewcommand{\tablename}{表}					% 改變表格標號文字為中文的「表」(預設為 Table)
\renewcommand{\figurename}{圖}				% 改變圖片標號文字為中文的「圖」(預設為 Figure)
\usepackage{booktabs}


% 設定顏色 see color Table: http://latexcolor.com
\usepackage{color}
\definecolor{slight}{gray}{0.9}				
\definecolor{airforceblue}{rgb}{0.36, 0.54, 0.66} 
\definecolor{arylideyellow}{rgb}{0.91, 0.84, 0.42}
\definecolor{babyblue}{rgb}{0.54, 0.81, 0.94}
\definecolor{cadmiumred}{rgb}{0.89, 0.0, 0.13}
\definecolor{coolblack}{rgb}{0.0, 0.18, 0.39}
\definecolor{beaublue}{rgb}{0.74, 0.83, 0.9}
\definecolor{beige}{rgb}{0.96, 0.96, 0.86}
\definecolor{bisque}{rgb}{1.0, 0.89, 0.77}
\definecolor{gray(x11gray)}{rgb}{0.75, 0.75, 0.75}
\definecolor{limegreen}{rgb}{0.2, 0.8, 0.2}
\definecolor{splashedwhite}{rgb}{1.0, 0.99, 1.0}

%---------------------------------------------------------------------
% 映出程式碼 \begin{lstlisting} 的內部設定
\lstset
{	language=[LaTeX]TeX,
    breaklines=true,
    %basicstyle=\tt\scriptsize,
    basicstyle=\tt\normalsize,
    keywordstyle=\color{blue},
    identifierstyle=\color{black},
    commentstyle=\color{limegreen}\itshape,
    stringstyle=\rmfamily,
    showstringspaces=false,
    %backgroundcolor=\color{splashedwhite},
    backgroundcolor=\color{slight},
    frame=single,							%default frame=none 
    rulecolor=\color{gray(x11gray)},
    framerule=0.4pt,							%expand outward 
    framesep=3pt,							%expand outward
    xleftmargin=3.4pt,		%to make the frame fits in the text area. 
    xrightmargin=3.4pt,		%to make the frame fits in the text area. 
    tabsize=2				%default :8 only influence the lstlisting and lstinline.
}

% 映出程式碼 \begin{lstlisting} 的內部設定 for Python codes
%\lstset{language=Python}
%\lstset{frame=lines}
%\lstset{basicstyle=\SCP\normalsize}
%\lstset{keywordstyle=\color{blue}}
%\lstset{commentstyle=\color{airforceblue}\itshape}
%\lstset{backgroundcolor=\color{beige}}

%%%數學式定義定理範例設定%%%
%%theroemstyle 需要使用的套件%%
\usepackage{amsthm}						
\theoremstyle{plain}

%%Definition獨立編號%%
\newtheorem{de}{定義}[section]
%%Theorem獨立編號%%
\newtheorem{thm}{{\MB 定理}}[section]		
%%lemma和theorem共同編號%%
\newtheorem{lemma}[thm]{Lemma}	
%%Example獨立編號%%
\newtheorem{ex}{{\A Example}}			

\newtheorem{cor}{Corollary}[section]		%not used here
\newtheorem{exercise}{EXERCISE}			%not used here
\newtheorem{re}{\emph{Result}}[section]	%not used here
\newtheorem{axiom}{AXIOM}				%not used here
\renewcommand{\proofname}{證明}			%not used here
\newcounter{quiz}						% start a simple and new counter
\setcounter{quiz}{1}						% start to count from 1
   % 使用自己維護的定義檔
%變更章節數字為中文數字
\geometry{a4paper,left=2.5cm,right=2.5cm,top=2cm,bottom=2cm}
\usepackage{fancyhdr}
\pagestyle{fancy}
\fancyhf{}
%\fancyhead[LE,RO]{overleaf}
%\fancyhead[RE,LO]{guides}
%\fancyhead[CE,CO]{\leftmark}
%\fancyhead[LE,RO]{\thepage}


\title{{\MB 機率分配與抽樣分配}}	% 使用設定的字型
\author{{\BB 郭翊萱 711133115}}				% 使用設定的小字體
\date{{\ST 2022.11 }} 			
 
\begin{document}
\maketitle
\fontsize{12}{22pt}\selectfont 
%\chapter{{\MB 機率分配與抽樣分配}}

統計學的基礎是機率,而我們需要更了解每個分配的基本性質才能更好的理解統計學,並做出正確的分析。因此在此文件中,我們首先將討論在改變參數時,幾種常見的離散分配與連續分配的機率密度圖與累積機率圖,接著,我們會利用抽樣分配來探討幾種分配之間的關係,並以直方圖、qqplot 圖、箱型圖與 edcf 圖進行呈現。最後,我們將以一個小專題來進行此節的收尾。

\section{Discrete Distributions}

在此節中,我們將介紹幾種常見的離散型分配,包含二項分配(Binomial Distribution)、超幾何分配(Hypergeometric Distribution)、幾何分配(Geometric Distribution)與卜瓦松分配 (Poisson Distribution)。首先我們先介紹二項分配。

\subsection{Binomial Distribution}

二項分配是 n 個獨立的試驗中成功次數的離散機率分布,每次成功的機率都設為p,而一次成功的分配則稱為伯努利分配。以下我們透過改變二項分配的參數來繪製其機率分布圖。
\begin{figure}[H]
    \centering
        \includegraphics[scale=0.5]{/Users/guoyixuan/Documents/vscodepython/Statistical Calculation/Homework3_Distribution/image_hw3/binom.eps}
    \caption{cdf and pdf of Binomial Distribution}
    \label{fig:Binomial Distribution}
\end{figure}

    以上圖 \ref{fig:Binomial Distribution} 即為二項分配的pmf圖與cdf圖。我們透過改變n (綠色 stem 圖)以及改變p (藍色 stem 圖)分別繪製二項分配的pmf 圖。先觀察綠色的  pmf 圖,我們令 n = 50, 70, 90, 110、p為0.7來繪製 pmf 圖,由圖 \ref{fig:Binomial Distribution} 可知,隨著n值變大,整個圖型會往右偏移,且會逐漸趨近於鐘型分配,似乎與常態分配趨近。接著我們觀察藍色的pmf圖,我們令 n 為100、p = 0.3, 0.6, 0.9來繪製機率圖,由圖 \ref{fig:Binomial Distribution} 同樣可以發現,隨著 p 增大,圖型也會逐漸向右移動,即期望值變大,且圖型的最高點也從0.08左右升至0.12左右。

\bigskip
\begin{lstlisting}
n = np.arange(50, 120, 20)
p = 0.7
x = np.linspace(20, 100, 50)
for i in n:
    y = binom.pmf(x, i, p)
    axes[0].stem(x, y, linefmt='g-', markerfmt='o', basefmt = 'C1--')
n = 100
p = np.arange(0.3, 0.9, 0.3)
for i in p:
    y = binom.pmf(x, n, i)
    axes[0].stem(x, y, linefmt='b:', markerfmt='o', basefmt = 'C1--')
\end{lstlisting}

以上即為繪製pmf圖所需之程式碼,其中 linefmt 可以改變 stem 圖的線條型狀以及線條顏色,markerfmt 則可以改變端點的形狀。

透過圖 \ref{fig:Binomial Distribution} 同樣可以觀察二項分配的 cdf 圖,我們繪製三組不同 (n, p) 情況下的 cdf 圖,包含 (20, 0.5), (30, 0.6) 以及 (40, 0.7),隨著 n 與 p 的值增加,cdf 圖會整體向右移動。

\textbf{Binomial Approximation}

在此我們以二項分配 B(100, 0.1) 與常態分配 N(10, 3) 舉例,由圖 \ref{fig:Binomial Approximation} 可知,二項分配在樣本數夠大且 $np > 5$ 時會趨近常態分配。

\begin{figure}[H]
    \centering
        \includegraphics[scale=0.5]{/Users/guoyixuan/Documents/vscodepython/Statistical Calculation/Homework3_Distribution/image_hw3/binom-norm.eps}
    \caption{Binomial Approximate to Normal Distribution}
    \label{fig:Binomial Approximation}
\end{figure}


\subsection{Hypergeometric Distribution}

超幾何分配 (Hypergeometric Distribution) 是統計學上的一種離散型分配,代表從有限的 M 個物件中抽取 N 個物件,並成功從其中指定的 N 個物件中抽出 x 個物件的機率(取出不放回)。

\begin{figure}[H]
    \centering
        \includegraphics[scale=0.6]{/Users/guoyixuan/Documents/vscodepython/Statistical Calculation/Homework3_Distribution/image_hw3/hyper.png}
    \caption{Binomial Approximate to Normal Distribution}
    \label{fig:Hypergeometric Distribution}
\end{figure}

我們透過改變 n 以及 N 來繪製 pmf 圖進行觀察,透過圖 \ref{fig:Hypergeometric Distribution} 可知,隨著 n 與 N 的值增加,超幾何分配的圖形也會逐漸向右偏移,即期望值變大,而且圖型的最高點會逐漸變小,從0.15左右下降至0.1左右。

\subsection{Geometric Distribution}
幾何分配 (Geometric Distribution) 可以用兩者方法來進行解釋,第一種解釋是 "在伯努利試驗中,得到一次成功所需的試驗次數(X)",另一種意思則為 "在得到第一次成功之前所經歷的失敗次數(Y=X-1)"。其 pmf 為:

\begin{equation}
P(X=k)=(1-p)^{k-1}p
\end{equation}

其中 $k = 1,2,3,...$

我們透過變更參數 p 來觀察幾何分配的 pmf 圖,透過觀察圖 \ref{fig:Geometric Distribution} 我們可知,我們設 p = 0.2, 0.4, 0.5, 0.7 來進行觀察,隨著 p 的值增大,pmf 圖的端點從0.2增至0.7,其尾部也從遞減至 $x=12$ 變成遞減至 $x=6$ 左右。

\begin{figure}[H]
    \centering
        \includegraphics[scale=0.6]{/Users/guoyixuan/Documents/vscodepython/Statistical Calculation/Homework3_Distribution/image_hw3/geom.eps}
    \caption{Geometric Distribution}
    \label{fig:Geometric Distribution}
\end{figure}


\subsection{Poisson Distribution}

卜瓦松分配 (Poisson Distribution) 是用於描述單位時間內隨機事件發生的次數的機率分配,其參數 $\lambda$ 為隨機事件發生次數的期望值,其機率質量函數則為:

\begin{equation}
P(X=k)=\frac{e^{-\lambda}\lambda^k}{k!}
\end{equation}

\begin{figure}[H]
    \centering
        \includegraphics[scale=0.6]{/Users/guoyixuan/Documents/vscodepython/Statistical Calculation/Homework3_Distribution/image_hw3/poisson.eps}
    \caption{Poisson Distribution}
    \label{fig:Poisson Distribution}
\end{figure}

我們透過改變參數 $\lambda$ 來觀察卜瓦松分配的 pmf 圖,由圖 \ref{fig:Poisson Distribution} 可知,我們設定參數 $\lambda=1, 4, 6, 8$,當 $\lambda$ 較小時,卜瓦松分配為右偏分配,而當 $\lambda$ 較大時,卜瓦松分配則為鐘型分配,可能趨近常態分配。

\textbf{Binomial Approximate to Poisson Distribution}

在數理統計課程中,我們可知當 n 趨近於無限大、p 趨近於0時二項分配會趨近於卜瓦松分配,因此在此處我們透過改變二項分配的參數 n, p 以及卜瓦松分配的參數 $\lambda$ 來觀察此趨近特性。我們設置 $\lambda=np$。

\begin{figure}[H]
    \centering
        \includegraphics[scale=0.6]{/Users/guoyixuan/Documents/vscodepython/Statistical Calculation/Homework3_Distribution/image_hw3/binom-poisson.eps}
    \caption{Binomial approximate to Poisson Distribution}
    \label{fig:Binomial approximate to Poisson Distribution}
\end{figure}

由圖 \ref{fig:Binomial approximate to Poisson Distribution} 的左上圖與右上圖可知,當 $n=20$、$p=0.05$ 時,二項分配趨近於卜瓦松分配,但當 $n=20$、$p=0.5$ 時,二項分配不會趨近於卜瓦松分配。

接著,根據圖 \ref{fig:Binomial approximate to Poisson Distribution} 的 左下圖與右下圖可知,當 n 夠大且 p 夠小時二項分配會趨近於卜瓦松分配,而當 n 夠大但 p 不夠小時,二項分配不會趨近於卜瓦松分配。下面即為部分程式碼示例。

\bigskip
\begin{lstlisting}
fig, axes = plt.subplots(2, 2, figsize=(10, 5))
x = np.linspace(1, 6, 50)

n, p = 10, 0.05
lamb = n*p
y = binom.pmf(x, n, p)
axes[1][0].stem(x, y, label="B(n={},p={})".format(n,p))
y1 = poisson.pmf(x, lamb)
axes[1][0].stem(x, y1, linefmt='g:', label="Poisson($\lambda$={})".format(lamb))
axes[1][1].set_xlabel("x")

plt.savefig(img_dir+"binom-poisson.eps", format="eps")
plt.show()
\end{lstlisting}

\textbf{Poisson Distribution Addiction}

在數理統計中,我們同樣學到當 $X \sim Poisson(\lambda_{1})$、$X \sim Poisson(\lambda_{2})$ 時,$X+Y \sim Poisson(\lambda_{1}+\lambda_{2})$,即卜瓦松分配具有加成性。因此在此處,我們將利用亂數產生 X 與 Y 的卜瓦松分配隨機變數,並與 $X+Y \sim Poisson(\lambda_{1}+\lambda_{2})$ 的 pmf圖進行比較。在此處中,我們繪製直方圖 (Histogram)、箱型圖 (Boxplot)、常態機率圖與 Empirical CDF 圖來進行觀察,並觀察如果更改亂數產生的樣本大小 (n) 是否會使結果發生改變。

\begin{figure}[H]
    \centering
        \includegraphics[scale=0.6]{/Users/guoyixuan/Documents/vscodepython/Statistical Calculation/Homework3_Distribution/image_hw3/sampling poisson n=80.eps}
    \caption{$X+Y \sim Poisson(\lambda_{1}+\lambda_{2})$, $n=80$}
    \label{fig:Poisson Addiction when n = 80}
\end{figure}

我們設定$\lambda=3$ 以及 $\lambda=4$ 來繪製圖形,圖 \ref{fig:Poisson Addiction when n = 80} 即為利用亂數產生圖形的結果。由此圖可知,雖然 qqplot 圖基本上貼合 $Poisson(\lambda_{1}+\lambda_{2})$ 的 pmf 圖,但從其直方圖與 ecdf 圖卻可以看出,似乎亂數產生的分配的 ecdf 與 $Poisson(\lambda_{1}+\lambda_{2})$ 的 cdf 圖 並未完全貼合。以下為部分繪圖程式碼。

\bigskip
\begin{lstlisting}
np.random.seed(seed=1294) #設定種子

n = 1000
lamb1 = 3
x1 = poisson.rvs(lamb1, size = n)
lamb2 = 4
x2 = poisson.rvs(lamb2, size = n)

##兩個亂數抽樣分配相加
x4 = x1+x2
bins = 10
axes[0][0].hist(x4, density=True, bins = bins, alpha=0.5, color="y")

###Poisson分配(lamb1+lamb2)
x3 = np.linspace(0, 20, 50)
lamb3 = lamb1+lamb2
y = poisson.pmf(x3, lamb3)
axes[0][0].stem(x3, y)

##ECDF##
x_sort = np.sort(x4)
F = np.arange(1 ,n+1) / n 
axes[1][1].plot(x_sort, F, lw =3, label = "Empirical CDF")

X = np.linspace(x_sort[0], x_sort[-1], 1000)
y = poisson.cdf(X, lamb3)
axes[1][1].plot(X, y, linestyle="--", lw = 3, label = "real cdf")
\end{lstlisting}

\begin{figure}[H]
    \centering
        \includegraphics[scale=0.6]{/Users/guoyixuan/Documents/vscodepython/Statistical Calculation/Homework3_Distribution/image_hw3/sampling poisson n=1000.eps}
    \caption{$X+Y \sim Poisson(\lambda_{1}+\lambda_{2})$, $n=1000$}
    \label{fig:Poisson Addiction when n = 1000}
\end{figure}

在圖 \ref{fig:Poisson Addiction when n = 1000} 中我們可以看出,與 $n=80$ 時不同的是,此時的qqplot圖完全貼合,且 ecdf 圖與真實的 $Poisson(\lambda_{1}+\lambda_{2})$ 也已完全貼近,由此可證。\\

\textbf{Sampling Poisson Distribution}

以下我們利用亂數抽取卜瓦松分配並繪製直方圖與 qqplot 圖,並觀察其與常態分配的差異以及樣本數改變可能導致的差異。在此我們以 Normal($\mu,\sigma$)作為比較,觀察 Poisson(2), Poisson(5) 與 Poisson(20) 三個分配。我們首先觀察在樣本數 n=20, n=100, n=1000 時的直方圖。

\begin{figure}[H]
    \centering
        \includegraphics[scale=0.5]{/Users/guoyixuan/Documents/vscodepython/Statistical Calculation/Homework3_Distribution/image_hw3/sampling-poinormal-hist-n=20.eps}
    \caption{Sampling Poisson Distribution, n=20}
    \label{fig:sampling-poinormal-hist-n=20}
\end{figure}

從圖 \ref{fig:sampling-poinormal-hist-n=20} 可知,在樣本數為20時,從直方圖中似乎無法觀察出甚麼特別的性質。接著,我們將樣本數增加至$n=100$,甚至是 $n=1000$,如下圖 \ref{fig:sampling-poinormal-hist-n=100} 與 \ref{fig:sampling-poinormal-hist-n=1000}。

\begin{figure}[H]
    \centering
        \includegraphics[scale=0.5]{/Users/guoyixuan/Documents/vscodepython/Statistical Calculation/Homework3_Distribution/image_hw3/sampling-poinormal-hist-n=100.eps}
    \caption{Sampling Poisson Distribution, n=100}
    \label{fig:sampling-poinormal-hist-n=100}
\end{figure}

\begin{figure}[H]
    \centering
        \includegraphics[scale=0.5]{/Users/guoyixuan/Documents/vscodepython/Statistical Calculation/Homework3_Distribution/image_hw3/sampling-poinormal-hist-n=1000.eps}
    \caption{Sampling Poisson Distribution, n=1000}
    \label{fig:sampling-poinormal-hist-n=1000}
\end{figure}

根據上圖 \ref{fig:sampling-poinormal-hist-n=100} 與 \ref{fig:sampling-poinormal-hist-n=1000} 我們可以發現,隨著樣本數變大以及卜瓦松分配的參數值變大,卜瓦松分配的直方圖會越來越趨近於鐘型分配的形狀。接著我們來觀察不同樣本數下卜瓦松分配的 qqplot 圖。

\begin{figure}[H]
    \centering
        \includegraphics[scale=0.5]{/Users/guoyixuan/Documents/vscodepython/Statistical Calculation/Homework3_Distribution/image_hw3/sampling-poinormal-qqplot-n=20.eps}
    \caption{Sampling Poisson Distribution, n=20}
    \label{fig:sampling-poinormal-qqplot-n=20}
\end{figure}

由圖 \ref{fig:sampling-poinormal-qqplot-n=20} 可觀察在樣本數為20的情況下,卜瓦松分配與常態分配的 qqplot 圖,在此圖中仍然可以觀察到卜瓦松分配的離散性質。接著,我們將樣本數增加至n = 100 甚至是 n = 1000 再重新繪製 qqplot 圖,如圖 \ref{fig:sampling-poinormal-qqplot-n=100} 與 \ref{fig:sampling-poinormal-qqplot-n=1000}。

\begin{figure}[H]
    \centering
        \includegraphics[scale=0.5]{/Users/guoyixuan/Documents/vscodepython/Statistical Calculation/Homework3_Distribution/image_hw3/sampling-poinormal-qqplot-n=100.eps}
    \caption{Sampling Poisson Distribution, n=100}
    \label{fig:sampling-poinormal-qqplot-n=100}
\end{figure}

\begin{figure}[H]
    \centering
        \includegraphics[scale=0.5]{/Users/guoyixuan/Documents/vscodepython/Statistical Calculation/Homework3_Distribution/image_hw3/sampling-poinormal-qqplot-n=1000.eps}
    \caption{Sampling Poisson Distribution, n=1000}
    \label{fig:sampling-poinormal-qqplot-n=1000}
\end{figure}

由上圖 \ref{fig:sampling-poinormal-qqplot-n=100} 與 \ref{fig:sampling-poinormal-qqplot-n=1000} 可觀察到,隨著樣本數增大,卜瓦松分配會與常態分配越來越貼近,此觀察到的結果與我們在數理統計課程中所了解到的性質一致。以下我們簡略呈現部分程式碼:

\bigskip
\begin{lstlisting}
mu = 4
sigma = 3
n = 1000
rv_norm = norm.rvs(loc = mu, scale = sigma, size = n)
lamb1 = 2
lamb2 = 5
lamb3 = 20
rv_poi1 = poisson.rvs(lamb1, size = n)
rv_poi2 = poisson.rvs(lamb2, size = n)
rv_poi3 = poisson.rvs(lamb3, size = n)
ax1.hist(rv_norm, bins = bins, density=True, color="r", alpha = 0.5)
stats.probplot(rv_poi1, dist = "norm", plot = ax2)
\end{lstlisting}


\section{Continuous Distributions}

\subsection{Chi-squared Distribution}

k 個獨立的標準常態分配變數的平方和服從自由度為 k 卡方分配,卡方分配也是一種特殊的伽瑪分配。其機率密度函數為:

\begin{equation}
f_k(x)=\frac{1}{2^{\frac{k}{2}}\Gamma{(\frac{k}{2})}}2^{\frac{k}{2}-1}e^{\frac{-x}{2}}
\end{equation}

\begin{figure}[H]
    \centering
        \includegraphics[scale=0.6]{/Users/guoyixuan/Documents/vscodepython/Statistical Calculation/Homework3_Distribution/image_hw3/chi-squared.eps}
    \caption{Chi-squared Distribution}
    \label{fig:Chi-squared Distribution}
\end{figure}

我們透過改變卡方分配的參數 df 來觀察卡方分配的性質。我們設置參數 $df=4, 6, 8,...,32$,由圖 \ref{fig:Chi-squared Distribution} 可知,隨著參數的值變大,卡方分配會漸漸從右偏分配變成趨近於常態分配。以下是部分的程式碼:

\bigskip
\begin{lstlisting}
img_dir = "D:/vscodepython/Statistical Calculation/Homework3_Distribution/image_hw3/"
xlim = [0, 50]
x = np.linspace(xlim[0], xlim[1], 1000) 
df = np.arange(4, 32, 2)
plt.figure()
plt.axis([xlim[0], xlim[1], 0, 0.2])
for i in df:
    y = chi2.pdf(x, i)
    plt.plot(x,y, lw=1, color='blue', alpha=0.4)
    
plt.yticks([0, 0.1, 0.2])
plt.savefig(img_dir+"chi-squared.eps", format="eps")
plt.show()
\end{lstlisting}


\textbf{Chi-squared Approximate to Normal and Gamma Distribution}

我們透過繪製卡方分配 $\chi^2(1000)$ 與 常態分配 Normal($\mu$,$\sigma$) 的 pdf 圖,以及繪製卡方分配 $\chi^2(10)$ 與伽瑪分配 $\Gamma(\frac{10}{2}, \beta)$ 的 pdf 圖來更好的理解卡方分配與常態分配的關係,並驗證若卡方分配的參數為 $v$,則此卡方分配與伽瑪分配 $\Gamma(\frac{v}{2}, \beta)$ 相等。

\begin{figure}[H]
    \centering
        \includegraphics[scale=0.6]{/Users/guoyixuan/Documents/vscodepython/Statistical Calculation/Homework3_Distribution/image_hw3/chi-gamma_norm.eps}
    \caption{Chi-squared Distribution Relationship}
    \label{Chi-squared Relationship}
\end{figure}

由上圖 \ref{Chi-squared Relationship} 可知,隨著卡方分配的參數增大,卡方分配會由右偏分配變成趨近於鐘型分配,且由左圖可知,卡方分配與常態分配的 pdf 圖相同,期望值為1000。另外,由其右圖可知,卡方分配$\chi^2(v)$ 與 $\Gamma(\frac{v}{2}, \beta)$ 確實相等。其部分相關程式碼如下:

\bigskip
\begin{lstlisting}
####卡方分配與常態分配####
df = 1000
x = np.linspace(790, 1210) 
y = chi2.pdf(x.reshape(-1,1), df=df)
mu = 1000
sigma = 45
X = np.linspace(mu-5*sigma, mu+5*sigma, 500) 
Y = norm.pdf(X, loc=mu, scale=sigma)
axes[0].plot(X, Y, lw=3,linestyle="--", label="Normal Distribution") 
axes[0].set_title("chi-normal distribution", fontsize = 'small')
axes[0].legend(fontsize = 'small', loc="upper left")

####卡方分配與伽瑪分配####
v = 10
df = v/2
x = np.linspace(0, 50, 100)
y1 = chi2.pdf(x.reshape(-1,1), df = v)
y2 = gamma.pdf(x, a = df, scale = 2)
\end{lstlisting}

\textbf{Sampling Chi-squared Distribution}

接著我們在此利用亂數產生服從常態分配的隨機變數,並期望證明標準常態分配的平方 $Z^2$ 與 $\chi^2(1)$ 相等。我們透過繪製直方圖、箱型圖、qqplot 圖與 ecdf 圖來觀察究竟兩分配是否相同。另外,我們亦透過改變生成亂數的樣本大小 n 來觀察是否樣本數大小會造成兩分配的差異。

\begin{figure}[H]
    \centering
        \includegraphics[scale=0.4]{/Users/guoyixuan/Documents/vscodepython/Statistical Calculation/Homework3_Distribution/image_hw3/sampling-chi-normal-n=100.eps}
    \caption{Chi-squared Distribution Relationship, n=100}
    \label{Chi-squared Distribution Relationship, n=100}
\end{figure}



根據圖 \ref{Chi-squared Distribution Relationship, n=100} 我們可以明顯發現,當樣本數達到100筆時,亂數產生的常態隨機變數的機率圖與卡方分配 $\chi^2(1)$ 並未完全貼合,無法證明標準常態分配的平方 $Z^2$ 與 $\chi^2(1)$ 相等。因此我們將樣本數增大至10000筆並重新觀察直方圖、箱型圖、qqplot圖與 ecdf 圖,結果如圖 \ref{Chi-squared Distribution Relationship, n=10000}。以下我們呈現部分繪圖之程式碼。

\bigskip
\begin{lstlisting}
n = 100
x = norm.rvs(loc = 0, scale = 1, size = n)
x = x**2
bins = 150

x_z = np.linspace(0, 10, 1000)
z = chi2.pdf(x_z, df = 1)

####boxplot####
boxprops = dict(linestyle = '--', linewidth = 3, color = 'darkgoldenrod')
flierprops = dict(marker='o', markerfacecolor = 'blue', markersize = 8, linestyle = 'none') 
labels = ["$\chi^2(\lambda)$"]
axes[0][1].boxplot(np.r_[x, z], boxprops = boxprops, flierprops = flierprops, labels = labels)
####QQplot####
stats.probplot(x, dist = "chi2", sparams=(df), plot=axes[1][0])
\end{lstlisting}

\begin{figure}[H]
    \centering
        \includegraphics[scale=0.5]{/Users/guoyixuan/Documents/vscodepython/Statistical Calculation/Homework3_Distribution/image_hw3/sampling-chi-normal-n=10000.eps}
    \caption{Chi-squared Distribution Relationship, n=10000}
    \label{Chi-squared Distribution Relationship, n=10000}
\end{figure}

由圖 \ref{Chi-squared Distribution Relationship, n=10000} 可知,當樣本數由 $n=100$ 增加至 $n=10000$ 筆時,標準常態分配的平方 $Z^2$ 與 $\chi^2(1)$ 相等。

\subsection{Exponential Distribution}

指數分配 (Exponential Distribution) 是一種連續機率分配,其用來表示獨立隨機事件發生所需的時間間隔。指數分配的機率密度函數為:

\begin{equation}
f(x;\lambda)=\lambda e^{-\lambda x}, x \geq 0
\end{equation}

其中 $\lambda$ 為分配的母數,即代表每單位時間發生數件的次數,$\beta$ 為比例母數,即代表該事件在每單位時間的發生率,可以利用 $\lambda=\frac{1}{\beta}$ 來代替 $\lambda$。指數分配的期望值是 $\frac{1}{\lambda}$,變異數則為 $\frac{1}{\lambda^2}$。接著我們透過改變參數 $\lambda$ 來觀察指數分配的 pdf 圖與 cdf 圖。

\begin{figure}[H]
    \centering
        \includegraphics[scale=0.5]{/Users/guoyixuan/Documents/vscodepython/Statistical Calculation/Homework3_Distribution/image_hw3/exp.eps}
    \caption{Exponential Distribution}
    \label{Exponential Distribution}
\end{figure}

我們設定參數 $\lambda=1, 2, 3$,根據圖 \ref{Exponential Distribution} 可知,隨著參數 $\lambda$ 的值增大,單位時間發生事件次數是0的次數變大,且 pdf 圖遞減的速度加快。以下我們呈現部分的程式碼:

\bigskip
\begin{lstlisting}
x1 = np.arange(0, 5, 0.1) #(15,)
scale = np.arange(1, 4) #(4,)
for i in scale:
    y1 = i * np.exp(-i*x1)
    axes[0].plot(x1, y1, lw=2, label="scale={}".format(i))
\end{lstlisting}

在此題繪圖時遇到一些問題。若我們使用 $expon.pdf$ 來生成 pdf 圖,則繪製出的圖形會有點怪異,如下圖 \ref{Exponential Distribution1},但目前尚未找到原因。部分程式碼亦呈現於下。

\begin{figure}[H]
    \centering
        \includegraphics[scale=0.5]{/Users/guoyixuan/Documents/vscodepython/Statistical Calculation/Homework3_Distribution/image_hw3/exp1.eps}
    \caption{Exponential Distribution1}
    \label{Exponential Distribution1}
\end{figure}

\bigskip
\begin{lstlisting}
x1 = np.arange(0, 5, 0.1) #(15,)
scale = np.arange(1, 4) #(4,)
for i in scale:
    y1 = expon.pdf(x1, i)
    axes[0].plot(x1, y1, lw=2, label="scale={}".format(i))
\end{lstlisting}

\subsection{Double Exponential Distribution}

雙指數分配 (Double Exponential Distribution),亦稱為拉普拉斯分配 (Laplace Distribution),此分配可以看作兩個平移指數分配背靠背拼接在一起,其機率密度函數為:

\begin{equation}
f(x|\mu, b)=\frac{1}{2b}exp(-\frac{\left | x-\mu \right|}{b})
\end{equation}

\begin{figure}[H]
    \centering
        \includegraphics[scale=0.6]{/Users/guoyixuan/Documents/vscodepython/Statistical Calculation/Homework3_Distribution/image_hw3/laplace.eps}
    \caption{Laplace Distribution}
    \label{Laplace Distribution}
\end{figure}

我們用函數 $laplace.pdf$ 進行繪圖。由圖 \ref{Laplace Distribution} 可知,當參數 $b$ 從1增加到4時,整個 pdf 圖會逐漸變寬,變異數變大且端點的值從0.5左右下降至0.1左右。$\mu$ 的改變則會讓圖形產生左移或右移,若 $\mu$ 值變大會往右移,變小則往左移。

\subsection{Gamma Distribution}

假設 $X_1,X_2,..X_n$ 為連續發生事件的等候時間,且這 n 次等候時間相互獨立,則 $Y=X_1+X_2+...+X_n$ 服從伽瑪分配 (Gamma($\alpha,\beta$)),其中 $\alpha=n$,而 $\beta$ 與 $\lambda$ 互為倒數,$\lambda$ 代表單位時間內事件的發生率。其機率密度函數為:

\begin{equation}
f(x)=\frac{x^{\alpha -1} \lambda^{-\alpha}e^{(-\lambda x)}}{\Gamma(\alpha)}, x > 0
\end{equation}

\bigskip
\begin{lstlisting}
x = np.linspace(0, 20, 1000)
k = np.arange(1, 13, 1.5)# 0 1.5 3 4.5 6 7.5
theta = np.arange(1, 5, 0.5) #0 0.5 1 1.5 2 2.5
param = np.vstack((k,theta))#(2,6)#矩陣
param = param.T #(6,2)
for i in range(8):
     y = gamma.pdf(x, param[i][0], param[i][1])
     plt.plot(x, y, lw=2, c="blue", alpha=0.5)
\end{lstlisting}

\begin{figure}[H]
    \centering
        \includegraphics[scale=0.5]{/Users/guoyixuan/Documents/vscodepython/Statistical Calculation/Homework3_Distribution/image_hw3/gamma.eps}
    \caption{Gamma Distribution}
    \label{Gamma Distribution}
\end{figure}

以上即為繪製伽瑪分配 pdf 圖與 cdf 圖的程式碼與結果。我們設置其參數 $\alpha=1,2.5,4,5.5,7,8.5,10,11.5$ 且 $\beta=1,1.5,2,2.5,3,3.5,4,4.5$ 共八組不同參數畫出深藍色的 pdf 圖,接著我們設置參數 $(\alpha,\beta)=(1,2),(1,5),(1,9)$ 來觀察改變參數 $\beta$ 造成的圖形變化。根據圖 \ref{Gamma Distribution},隨著 $\alpha$ 值與 $\beta$ 值變大,伽瑪分配會逐漸從右偏分配變成鐘型分配,此性質與卡方分配相近,而若我們僅僅增大參數 $\beta$ 的值,分配則會逐漸向右偏移。

\textbf{Gamma Distribution Addiction}

在了解了伽瑪分配的基本性質後,我們接著利用亂數產生伽瑪分配來嘗試證明伽瑪分配的可加性,即當 $X \sim \Gamma(\alpha_1,\beta_1)$、$Y \sim \Gamma(\alpha_2,\beta_2)$ 時,$X+Y \sim \Gamma(\alpha_1+\alpha_2,\beta_1+\beta_2)$。我們將亂數產生 X, Y 兩個獨立分配並與 X+Y 分配共同繪製直方圖、qqplot 圖與 ecdf 圖,並更改樣本數來觀察樣本數可能造成的差異。

\begin{figure}[H]
    \centering
        \includegraphics[width=0.9\textwidth, height=0.5\textwidth]{/Users/guoyixuan/Documents/vscodepython/Statistical Calculation/Homework3_Distribution/image_hw3/sampling-gamma-n=100.eps}
    \caption{Gamma Distribution Addiction, n=100}
    \label{sampling-gamma-n=100}
\end{figure}

上圖 \ref{sampling-gamma-n=100} 即為當樣本數為100時,抽樣分配$X \sim \Gamma(1,10)$, $Y \sim \Gamma(10,1)$ 與 $X+Y \sim \Gamma(1+10,10+1)$ 的擬合結果。由圖可知,三種圖形得到的結果都是兩者並不是非常貼近,無法證明伽瑪分配的可加性,因此我們嘗試將樣本數提高到n = 1000再重新繪製三種圖形。以下提供部分程式碼:

\bigskip
\begin{lstlisting}
alpha1 = 1
beta1 = 10
beta2 = 1
alpha2 = 10
n = 100
x1 = gamma.rvs(alpha1, beta1, size=n)
x2 = gamma.rvs(alpha2, beta2, size=n)
x1 = np.sort(x1)
x2 = np.sort(x2)
x3 = x1+x2
axes[0].hist(x3, bins=100, density=True)
x4 = np.linspace(0, 35, 1000)
y = gamma.pdf(x4, alpha1+alpha2, beta1+beta2)
axes[0].plot(x4, y)
####ecdf####
x_sort = np.sort(x3)
F = np.arange(1 ,n+1) / n 
axes[2].plot(x_sort, F, lw =3, label="Empirical CDF")
X = np.linspace(x_sort[0], x_sort[-1], 1000)
y = gamma.cdf(X, alpha1+alpha2, beta1+beta2)
axes[2].plot(X, y, linestyle="--", lw = 3, label="real CDF")
\end{lstlisting}

\begin{figure}[H]
    \centering
        \includegraphics[width=0.9\textwidth, height=0.5\textwidth]{/Users/guoyixuan/Documents/vscodepython/Statistical Calculation/Homework3_Distribution/image_hw3/sampling-gamma-n=1000.eps}
    \caption{Gamma Distribution Addiction, n=1000}
    \label{sampling-gamma-n=1000}
\end{figure}

由圖 \ref{sampling-gamma-n=1000} 可知,亂數產生的$X \sim \Gamma(\alpha_1,\beta_1)$、$Y \sim \Gamma(\alpha_2,\beta_2)$ 與分配 $X+Y \sim \Gamma(\alpha_1+\alpha_2,\beta_1+\beta_2)$ 已經幾乎擬合,由此可證伽瑪分配的可加性。

\textbf{Sampling Gamma Distribution}

以下我們亂數抽取服從伽瑪分配的隨機變數並繪製直方圖與 qqplot 圖,並觀察其與常態分配的差異以及改變樣本數可能導致的差異。我們將觀察 Norma(4,3) 與 $\Gamma(1,31)$、$\Gamma(31,1)$ 與 $\Gamma(50,50)$ 三個分配在樣本數 n=50, n=1000時的圖形。

\begin{figure}[H]
    \centering
        \includegraphics[scale=0.5]{/Users/guoyixuan/Documents/vscodepython/Statistical Calculation/Homework3_Distribution/image_hw3/sampling-gammanormal-hist-n=50.eps}
    \caption{Histogram of Sampling Gamma Distribution, n=50}
    \label{sampling-gammanormal-hist-n=50}
\end{figure}

圖 \ref{sampling-gammanormal-hist-n=50} 呈現了在樣本數 n=50 時的直方圖,從此直方圖可以看出伽瑪分配的右偏與左偏等特性。接著我們將抽取的樣本數增加至 n=1000,可得圖 \ref{sampling-gammanormal-hist-n=1000}。

\begin{figure}[H]
    \centering
        \includegraphics[scale=0.5]{/Users/guoyixuan/Documents/vscodepython/Statistical Calculation/Homework3_Distribution/image_hw3/sampling-gammanormal-hist-n=1000.eps}
    \caption{Histogram of Sampling Gamma Distribution, n=1000}
    \label{sampling-gammanormal-hist-n=1000}
\end{figure}

從圖 \ref{sampling-gammanormal-hist-n=1000} 我們可以看出,當樣本數增加至 n=1000 時,$\Gamma(50,50)$、$\Gamma(31,1)$ 與常態分配的形狀趨近。接著我們來觀察Norma(4,3) 與 $\Gamma(1,31)$、$\Gamma(31,1)$ 與 $\Gamma(50,50)$ 三個分配的 qqplot 圖。

\begin{figure}[H]
    \centering
        \includegraphics[scale=0.5]{/Users/guoyixuan/Documents/vscodepython/Statistical Calculation/Homework3_Distribution/image_hw3/sampling-gammanormal-qqplot-n=50.eps}
    \caption{QQplot of Sampling Gamma Distribution, n=50}
    \label{sampling-gammanormal-qqplot-n=50}
\end{figure}

\begin{figure}[H]
    \centering
        \includegraphics[scale=0.5]{/Users/guoyixuan/Documents/vscodepython/Statistical Calculation/Homework3_Distribution/image_hw3/sampling-gammanormal-qqplot-n=1000.eps}
    \caption{QQplot of Sampling Gamma Distribution, n=1000}
    \label{sampling-gammanormal-qqplot-n=1000}
\end{figure}

\newpage

由圖 \ref{sampling-gammanormal-qqplot-n=50} 與 \ref{sampling-gammanormal-qqplot-n=1000} 我們可以看出,無論樣本數有多大,$\Gamma(1,31)$ 都不會貼近常態分配,而 $\Gamma(31,1)$ 與 $\Gamma(50,50)$ 在樣本數僅有50時還不與常態分配貼近,但當樣本數達到1000時,兩個分配都會貼近常態分配。

\subsection{Beta Distribution}

貝塔分配 (Beta Distribution) 是一組定義在區間 (0,1)上的連續機率分配,其機率密度函數為:

\begin{equation}
f(x;\alpha,\beta)=\frac{x^{\alpha-1}(1-x)^{\beta-1}}{\int_{0}^{1}u^{\alpha-1}(1-u)^{\beta-1}du}=\frac{\Gamma(\alpha+\beta)}{\Gamma(\alpha)\Gamma(\beta)}x^{\alpha-1}(1-x)^{\beta-1}
\end{equation}

另外,貝塔分配的期望值為 $\frac{\alpha}{\alpha+\beta}$、變異數為 $\frac{\alpha \beta}{(\alpha+\beta)^2 (\alpha+\beta+1)}$。

\begin{figure}[H]
    \centering
        \includegraphics[scale=0.6]{/Users/guoyixuan/Documents/vscodepython/Statistical Calculation/Homework3_Distribution/image_hw3/beta.eps}
    \caption{Beta Distribution}
    \label{Beta Distribution}
\end{figure}

接著我們設置不同的 $\alpha,\beta$ 來理解貝塔分配的機率密度函數。在下圖 \ref{Beta Distribution} 中,左上圖我們設置參數 $a > b$,a 的值介在1到30之間,b的值介在31到60之間。右上圖我們則設置參數為 $a < b$,即a 的值介在31到60之間,b的值介在1到30之間;在左下圖中,我們設置 a 與 b 的值相同,且介在1到30之間;在右下圖中,我們則繪製給定 $a=15$,b 介在31到60之間的貝塔分配 pdf 圖。

由圖可知,在 $a>b$ 時,貝塔分配為右偏分配、在 $a<b$ 時,貝塔分配為左偏分配,而在 $a=b$ 時,貝塔分配則為鐘型分配。另外,在固定 a 的情況下,若 b 值增加時 ($a<b$),貝塔分配會逐漸變成左偏分配。以下我們呈現部分程式碼:

\bigskip
\begin{lstlisting}
a = np.arange(1, 30)
b = np.arange(31,60)
x = np.linspace(0, 1, 200) #向量
Y = beta.pdf(x.reshape(-1,1), a , b) #矩陣
axes[0][0].plot(x, Y, lw=2, c="g", alpha=0.5)
axes[0][0].set_xlim(0, 0.5)
\end{lstlisting}

\textbf{Sampling Beta Distribution}

接著,我們利用亂數產生抽樣分配 $Beta(15,30),Beta(30,30)$ 以及 $Beta(30,15)$,並繪製樣本數大小分別為$n=200, n=1000$ 的直方圖。

\begin{figure}[H]
    \centering
        \includegraphics[scale=0.6]{/Users/guoyixuan/Documents/vscodepython/Statistical Calculation/Homework3_Distribution/image_hw3/beta-hist.eps}
    \caption{Histogram of Sampling Beta Distribution}
    \label{beta-hist}
\end{figure}

在 \ref{beta-hist} 的左圖為樣本數 $n=200$ 的直方圖,右圖則為樣本數 $n=1000$ 的直方圖。由圖可知,樣本數夠大時,無論 $a<b,a=b$ 還是 $a>b$,貝塔分配的機率密度圖都會趨近於鐘型分配。接著,我們繪製在樣本數為1000時, $Beta(15,30),Beta(30,30)$ 以及 $Beta(30,15)$ 的 qqplot圖。

\begin{figure}[H]
    \centering
        \includegraphics[scale=0.6]{/Users/guoyixuan/Documents/vscodepython/Statistical Calculation/Homework3_Distribution/image_hw3/beta-qqplot.eps}
    \caption{QQplot of Sampling Beta Distribution}
    \label{beta-qqplot}
\end{figure}

由圖 \ref{beta-qqplot} 可知,亂數產生的服從貝塔分配的隨機變數會幾乎貼合紅色的線,即在樣本數夠大時,無論參數的值為何,貝塔分配都會趨近於常態分配。最後,我們亦繪製不同樣本數時的 ecdf 圖,如下圖 \ref{beta-ecdf-n change}。

\begin{figure}[H]
    \centering
        \includegraphics[scale=0.7]{/Users/guoyixuan/Documents/vscodepython/Statistical Calculation/Homework3_Distribution/image_hw3/beta-ecdf-n change.eps}
    \caption{ecdf of Sampling Beta Distribution}
    \label{beta-ecdf-n change}
\end{figure}

由圖 \ref{beta-ecdf-n change} 可知,當 $n=20$ 時,亂數抽取的抽樣分配與貝塔分配的 cdf 不完全相同,而在樣本數 $n=500$ 時,兩者則幾乎完全貼近。以下為其部分程式碼:

\bigskip
\begin{lstlisting}
n = 500
x1 = beta.rvs(a1, b1, size=n)
x2 = beta.rvs(a2, b2, size=n)
x3 = beta.rvs(a3, b3, size=n)
x_sort1 = np.sort(x1)
F = np.arange(1 ,n+1) / n 
axes[0].plot(x_sort1, F, lw =3)
x_sort2 = np.sort(x2)
x_sort3 = np.sort(x3)

x = np.linspace(0, 1 ,1000)
y1 = beta.cdf(x.reshape(-1, 1), a1, b1)
y2 = beta.cdf(x, a2, b2)
y3 = beta.cdf(x, a3, b3)
\end{lstlisting}

\subsection{Cauchy Distribution}

柯西分配 (Cauchy Distribution) 是一種連續機率分配,其機率密度函數為:

\begin{equation}
f(x;x_0,\gamma)=\frac{1}{\pi \gamma [1+(\frac{x-x_0}{\gamma})^2]}=\frac{1}{\pi}[\frac{\gamma}{(x-x_0)^2+\gamma^2}]
\end{equation}

在 $x_0=0$ 且 $\gamma=1$ 的特例為標準柯西分配,其機率密度函數為:

\begin{equation}
f(x;0,1)=\frac{1}{\pi (1+x^2)}
\end{equation}

\begin{figure}[H]
    \centering
        \includegraphics[scale=0.6]{/Users/guoyixuan/Documents/vscodepython/Statistical Calculation/Homework3_Distribution/image_hw3/cauchy.eps}
    \caption{Cauchy Distribution}
    \label{cauchy}
\end{figure}

在圖 \ref{cauchy} 中,我們透過改變參數 $x_0$ 與 $b$ 的值來觀察柯西分配的 pdf 與 cdf 圖。由其左圖可知,改變 $x_0$ 會改變其分配的位置,改變 $b$ 的值則會改變圖形的形狀,隨著 $b$ 值變大,分配的端點會從0.3左右變成0.05左右,而其 cdf 圖同樣會隨著 $x_0, b$ 的改變而變化。

\subsection{Normal Distribution}

\begin{figure}[H]
    \centering
        \includegraphics[scale=0.6]{/Users/guoyixuan/Documents/vscodepython/Statistical Calculation/Homework3_Distribution/image_hw3/normal.eps}
    \caption{Normal Distribution}
    \label{normal}
\end{figure}

上圖 \ref{normal} 即為常態分配改變參數所得之機率密度圖,由圖可知,$\mu$ 的改變會使分配平移,$\sigma$ 的改變則會使分配的圖形改變。以下我們呈現其部分的程式碼:

\bigskip
\begin{lstlisting}
mu = 0
sigma = np.arange(1,7)
xlim = [mu - 5 * sigma.max(), mu + 5 * sigma.max()]
x = np.linspace(xlim[0], xlim[1], 1000)
Y = norm.pdf(x.reshape(-1,1), loc=mu, scale=sigma)

plt.plot(x, Y, label = ["$\mu=0$, $\sigma$={}".format(i) for i in sigma], lw=2)
\end{lstlisting}

\subsection{t-distribution}

司徒頓 t 分配 (Student's t-distribution) 是用以根據小樣本來估計母體呈常態分配且標準差未知的期望值,若母體標準差已知或樣本數夠大時則用常態分配進行估計。t 分配的機率密度函數為:

\begin{equation}
f(t)=\frac{\Gamma{\frac{v+1}{2}}}{\sqrt{v\pi}\Gamma(\frac{v}{2})}(1+\frac{t^2}{v})^{\frac{-(v+1)}{2}}
\end{equation}

\begin{figure}[H]
    \centering
        \includegraphics[scale=0.5]{/Users/guoyixuan/Documents/vscodepython/Statistical Calculation/Homework3_Distribution/image_hw3/t-normal.eps}
    \caption{t Distribution}
    \label{t-normal}
\end{figure}

由圖 \ref{t-normal} 我們可以發現 t 分配與標準常態分配都是鐘型分配,且在 t 分配的自由度夠大時,t 分配會與標準常態分配相同。以下呈現部分程式碼:

\bigskip
\begin{lstlisting}
xlim = [-6, 6]
x = np.linspace(xlim[0], xlim[1], 1000)
df = np.r_[np.arange(0.1, 1, 0.1), np.arange(1, 30)]
plt.figure()
plt.axis([xlim[0], xlim[1], 0, 0.2])
for i in df:
    y=t.pdf(x, i)
    plt.plot(x,y, lw=1, color='blue', alpha=0.3)
\end{lstlisting}

\subsection{F-distribution}
定義隨機變數 X 有母數為 $d_1$ 與 $d_2$ 的 F 分配,寫作 $X \sim F(d_1,d_2)$,對於實數 $x \geq 0$,其機率密度函數為:

\begin{equation}
f(x;d_1,d_2)=\frac{\sqrt{\frac{(d_1 x)^{d_1} d_2^{d_2}}{(d_1 x+d_2)^{d_1+d_2}}}}{x B(\frac{d_1}{2},\frac{d_2}{2})}
\end{equation}

\begin{figure}[H]
    \centering
        \includegraphics[scale=0.6]{/Users/guoyixuan/Documents/vscodepython/Statistical Calculation/Homework3_Distribution/image_hw3/f.eps}
    \caption{F Distribution}
    \label{F}
\end{figure}

我們設置參數$d_1=1,2,3,4$ 且 $d_2=1,2,3,4$ 來繪製 F 分配的 pdf 圖與 cdf 圖,可得到圖 \ref{F} 中的實線圖,另外,我們亦設置參數 $d_1=10,30,50,70,90$ 與 $d_2=60$ 來繪製 pdf 圖與 cdf 圖,可得到圖 \ref{F} 中的虛線圖。

\begin{lstlisting}
x = np.linspace(0, 5, 1000)
gamma1 = np.arange(1, 5)
gamma2 = np.arange(1, 5)
param = np.vstack((gamma1, gamma2))
param = param.T
for i in range(4):
    y = f.pdf(x, param[i][0], param[i][1])
    plt.plot(x, y, lw=2)

\end{lstlisting}

\newpage

\section{Project}

給定四個數字 (2, 4, 9, 12)。從這四個數字中隨機抽取四個數字(取後放回)並計算其平均數。假設隨機變數 Y 代表這四個數字的平均數。請繪製隨機變數 Y 的 PMF。本題可以直接計算每個平均數的機率,但在此請使用隨機抽樣的方式,估計出這些機率值。其中抽樣的次數可以高至百萬以上。

\bigskip
\begin{figure}[H]
    \centering
        \includegraphics[scale=0.6]{/Users/guoyixuan/Documents/vscodepython/Statistical Calculation/Homework3_Distribution/image_hw3/sampling.eps}
    \caption{Sampling}
    \label{sampling}
\end{figure}

以上圖 \ref{sampling} 即為此抽樣的結果。其部分程式碼亦呈現如下:

\bigskip
\begin{lstlisting}
np.random.seed(seed=1294)
x = np.array([2, 4, 9, 12])
number = 10000
y = np.zeros(number)
for i in np.arange(len(y)):
    rc = np.random.choice(x, 4, replace=True)
    y[i] = rc.mean()
#plt.hist(y, bins= 50, density=True)#error

weight = np.ones_like(y)/len(y)
plt.hist(y, bins=50, weights=weight)
\end{lstlisting}

\section{Conclusion}

在本文中,我們成功利用各種圖形來更深入了解各個常用的機率分配的特性,望以後在對分配的特性產生疑問時,能直接利用此文件得到解答,也能更快速的利用 Python 來理解其他本文未提及之分配的特性。

\end{document}