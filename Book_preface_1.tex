\addcontentsline{toc}{chapter}{序}
\chapter*{序}

統計學在許多人心中或許是一門玄學,它基於機率用合理的猜去做合理的推論,但是人們往往對於如何去猜沒有很清晰的認知,許多人會認為統計學就是一門詐騙術,或在政治中操控民調結果,或在統計新冠肺炎數據時用似是而非的方式來引導人們的認知,而這都是基於對陌生事物的恐懼感。\\

人類往往習慣先看見再相信,而對未知事物的恐懼又會阻撓人類睜開雙眼,但是21世紀的人類最寶貴的資產或許就是想睜眼就能看見,而不似前人頂著狂風摸著黑暗前行,一昧的逃避與拒絕了解只會憑白浪費先人與世界給予我們的饋贈,人的一生短短數十年,了解這個世界的邏輯並熱愛這個世界數十年的人生或許才不會活著而像是沒有活著,當你不能善用這個世界給你的禮物而是閉著雙眼生活的時候。

這本書希望能藉由一個單元又一個單元去慢慢引導讀者對於統計學與數學的認知,希望能逐漸引導人們消彌對數學的恐懼以及對統計學的誤解,這本書將利用程式語言來幫助人們了解統計學這片廣袤的土地,利用各種各樣的模擬實驗來幫助讀者一窺其中的有趣之處。本書將首先介紹這本書的製作方式,即利用 \LaTeX 來協助所有的排版、目錄建立以及書本修訂,希望能幫助讀者未來有機會使用到這個工具時能藉由這本書瞭解大部分常用的指令。

第二章與第三章開始則將利用正當風靡的程式語言 Python 來幫助讀者大致了解一些基本的函數圖形以及 Python 中的套件 Matplotlib 的基本使用方式,並介紹多種統計學的基礎機率分配以及這些分配的基本性質,希望能幫助讀者對分配有一些基本的概念。

第四章到第七章則將為讀者介紹七種常用的機器學習方法,機器學習主要有兩個常見的領域,一個是建立變數間的函數利用函數做選擇題,也就是分類問題,另一種則是預測問題,也就是建立函數來做計算題,本書將介紹基本的七種方法,包含簡單線性迴歸 ( Simple Linear Regression )、加廣迴歸 ( Augmented Regression )、羅吉斯迴歸( Logistic Regression )、線性判別分析 ( Linear Discrimunant Analysis )、二次判別分析 ( Quadratic Discriminant Analysis )、K-近鄰演算法 ( KNN ) 以及深度學習領域的類神經網路模型 ( ANN ),並利用統計模擬的方式一一建立模擬資料來檢視這些方法的訓練誤差與測試誤差,希望能幫助讀者對這些方法的理論以及實際應用有更深入的了解,並能在閱讀完本書後能降低一些對未知事物的恐懼甚至嘗試去喜歡它熱愛它,了解這個世界更多一些從來都沒有壞處只有好處,望共勉。

\begin{flushright}
    郭翊萱
    \par\vspace*{-2pt}\hfill 2023年1月於台北大學
\end{flushright}
