%\input{preamble_CJK}   % 使用自己維護的定義檔
%%-----------------------------------------------------------------------------------------------------------------------
%% 文章開始
%\title{ \LaTeX{\MB 的數學符號與方程式}}
%\author{{\SM 汪群超}}
%\date{{\TT \today}} 	 
%\begin{document}
%\maketitle
%\fontsize{12}{22 pt}\selectfont
\chapter{ \LaTeX{\MB 的數學符號與方程式}}
本文將常見的數學符號與方程式以 \LaTeX 編排展示,希望降低使用  \LaTeX 編輯數學式的門檻,快速得到  \LaTeX 為人稱頌的優美數學式。不但為初學者提供編輯的概念與方法,也作為未來的文件編輯的參照樣本。本文內容參考吳聰敏老師專書「\cw{} 排版系統」、\footnote{前往 \cw{} 官方網站 http://homepage.ntu.edu.tw/$\sim$ntut019/cwtex/cwtex.html 下載《cwTeX 使用手冊(PDF)》。}學生的作品、及作者平日編輯講義時所發現具代表性的數學方程式。

\section{數式環境}
數學式可能以兩種型式出現,一是隨文數式(In-text Formula),是夾在文章段落中的數學式;譬如,當 $\alpha=2$ 時, $\alpha^3=8$ 。另一種是數學式自成一行或一個段落,我們稱之為展示數式(Display Formula),譬如

$$\int_{-2}^{1} f(x)\;dx$$

輸入數學式時,有兩個地方需要特別注意:
\begin{itemize}
\item 隨文數式前後請留一空格,才不會顯得擁擠。
\item 展示數式上下不須多留一空行, \LaTeX\ 會自行調整間距。
\end{itemize}

\section{符號}
數字與普通運算符號可直接由鍵盤上鍵入。譬如,下列符號可以直接由鍵盤鍵入:

        \begin{center}
         $  + \;-\; =\; <\; > \;/ \;:\; !\;\; |\; \;[\;\; ] \;(\; )$\\
        \end{center}
要注意的是, 左右大括號 $\{$ $\}$ 在 \LaTeX 中有特殊用途。欲排版左大括號, 需加上反斜線,指令為 $\backslash\{$ ,右大括號之指令為 $\backslash\}$ 。排版展示數式有以下四種方法可以達到目的:
        \begin{center}
$\backslash$begin$\{$equation$\}$ ... $\backslash$end$\{$equation$\}$\\
$\backslash$begin$\{$displaymath$\}$ ... $\backslash$end$\{$displaymath$\}$\\
$\backslash$[ ... $\backslash$]\\
\$\$ ... \$\$
        \end{center}
除第一種方式外,其餘將不對數學式子進行編號。數式內若要排版文字時,必須置於 $\backslash$mbox 指令內,否則將被視為數學符號(變為斜體),譬如,

$$f(x)=x^2-3x+1 \mbox{, where}  -2 \leq x \leq 2$$

\section{常見的數學式}
本節列舉一些常見的數學式作為練習與未來使用的參考,每個函數都有其特別之處,請仔細觀察研究。讀者可以依此為基礎,在往後的寫作過程中,逐漸累積更多有特殊型態的或符號的數學式,只要這裡出現過的,參照原始檔一定寫得出來。

\subsection{函數}
挑幾個機率分配函數做示範:

\textbf{Binomial}: 
$$f(x)={n\choose x}p^x(1-p)^{1-x}, \;\; x=0,1,2,\cdots,n$$ 

\textbf{Poisson}: 
$$f(x)=\frac{e^{-\lambda}\lambda^x}{x!}, \;\;  x=0,1,2,\cdots$$ 

\textbf{Gamma}: 
$$f(x)=\frac{1}{\Gamma(\alpha)\beta^\alpha}x^{\alpha-1}e^{-\frac{x}{\beta}}, \;\; x\geq 0$$

\textbf{Normal}: 
$$f(x)=\frac{1}{\sigma\sqrt{2\pi}}e^{-\frac{(x-\mu)^2}{2\sigma^2}}, \;\;  -\infty < x < \infty $$

\bigskip
積分式與方程式編號:
  
  \begin{equation}\label{gamma}%.................label後的名稱自訂,代表該方程式
  \int^\infty_0 x^{\alpha-1}e^{-\lambda x} dx = \frac{\Gamma(\alpha)}{\lambda^{\alpha}}
  \end{equation}
  
方程式  (\ref{gamma}) 是廣義 $\Gamma$ 積分。\footnote{這裡利用方程式標籤(label)來引用方程式,編號將自動更新。}\\
  
 開根號(開立方根):
  
  $$f(x)=\sqrt[3]{\frac {\displaystyle 4-x^{3}}{\displaystyle 1+x^{2}}}$$
  
 微分與極限(注意大刮號的使用):
  
  $$f'(x)=\frac{df(x)}{dx}=\lim_{h\rightarrow 0}\left(\frac{f(x+h)-f(x)}{h}\right)$$
  
數學中的括號隨著其涵蓋內容的多寡(層次)與長相(分數),其大小必須調整恰當,如上式的兩種大小不同的括號「$( \cdot)$ 」。外圍較大地括號使用 $\backslash$  left$($ 與 $\backslash$  right$)$ 令編譯器依需求自動調整為適當大小。另外,也可以手動控制括號、的大小,如

 $$ \bigg(\; \big( \;(\;\;\;) \;\big) \;\bigg) \;,\; \bigg[ \;\big[ \;[\;\;\;]\; \big]\; \bigg]\;,\; \bigg\{ \;\big\{ \;\{\;\;\;\} \;\big\} \;\bigg\}$$ 
  
  
  上下限的使用:
  
  $$\int_a^b f(x) dx \approx \lim_{n\rightarrow \infty}\sum_{k=1}^n f(x_k)\triangle x_k$$
  
  最佳化問題(向量使用粗體來表示):
  
  $$\max_{\mathbf{u},\mathbf{u}^T\mathbf{u}=1} \mathbf{u}^T\Sigma_X\mathbf{u}$$
  
  其他符號:
  $$\mathbf{e}=\mathbf{x}-\mathbf{x}_q=(I-P)\mathbf{x} \in V^{\perp}, \mbox{where}\; V\oplus V^{\perp}=\mathbf{R}^p $$

 上述的方程式有許多地方需要做「下標」(subscript)與上標(supscript),不管是在積分的上下界或是 $\Sigma$ 的上下範圍,或只是變數的下標與次方、、、等,做法都是用  $\_$ 做下標,用 $\wedge$ 製作上標。當上下標只有一個符號或字母時,可以不加括號,否則必須以括號涵蓋。
 
\subsection{矩陣與行列式}
矩陣或有規則排列的數學式或組合很常見,以下列舉幾種模式,請特別注意其使用的標籤及一些需要注意的小地方。譬如,\footnote{這裏的項目符號不是預設的 1, 2, ...,改用 a, b...的編號方式。}
\begin{enumerate}[a)]
  \item 矩陣的左右括號需個別加上。
  \item 直行各項之間是以 $\&$ 區隔。
  \item 除最後一列外,每列之末則加上換列指令 $\backslash\backslash$。
  \item 使用 {\A array} 指令時,須加上選項以控制每一直行內各數字或符號要居中排列、靠左或靠右。
\end{enumerate}

範例與注意事項:
\begin{enumerate}
  \item 左右方框括號的使用及各直行的對齊方式:
        $$ A = \left[
            \begin{array}{clr}
                a+b & mnop  & xy \\
                a+b & pn    & yz \\
                b+c & mp    & xyz
            \end{array} \right] $$

  \item 左右圓框刮號的使用及各式點狀:
        $$ A=\left(
            \begin{array}{cccc}
                a_{11} 	& a_{12} & \cdots 	& a_{1n}\\
                a_{21} 	& a_{22} & \cdots 	& a_{2n}\\
                \vdots 	& \vdots & \ddots	& \vdots\\
                a_{n1} 	& a_{n2} & \cdots 	& a_{nn}
            \end{array} \right) $$

  \item 排列整齊的符號:
        $$ \begin{array}{clr}\\
            a+b+c   & m+n 	& xy \\
            a+b     & p+n 	& yz \\
            b+c     & m-n 	& xz
        \end{array} $$

    \item 等號對齊的函數組合(不編號)
        \begin{eqnarray*}
          b_1 &=& d_1+c_1 \\
          a_2 &=& c_2+e_2
        \end{eqnarray*}

    \item 等號對齊的函數組合(編號在最後一行)
        \begin{eqnarray}
\nonumber b_1 &=& d_1+c_1 \\
          a_2 &=& c_2+e_2
        \end{eqnarray}

    \item 使用套件 {\A amsmath} 的指令 {\A align}(控制編號在第一行)
        \begin{align}
            b_1 &= d_1+c_1\\
            a_2 &= c_2+e_2 \notag
        \end{align}

    \item 兩組數學式分別對齊
    \begin{align}
        \alpha_1 &= \beta_1+\gamma_1+\delta_1, &a_1 &= b_1+c_1\\
        \alpha_2 &= \beta_2+\gamma_2+\delta_2, &a_2 &= b_2+c_2
    \end{align}

    \item 編號在中間({\A split} 指令環境)
        \begin{equation}
            \begin{split}
                \alpha_1 &= \beta_1+\gamma_1\\
                \alpha_2 &= \beta_2+\gamma_2
            \end{split}
        \end{equation}
    \item 只是居中對齊的數學式組(環境指令 {\A gather})
        \begin{gather}
        \alpha_1 + \beta_1\notag\\
        \alpha_2 + \beta_2 + \gamma_2\notag
        \end{gather}

    \item 長數學式的表達(注意第二行加號的位置)
        \begin{align}
            y  	&= x_1 + x_2 + x_3 \notag\\
                	&\quad + x_4 + x_5
        \end{align}
\end{enumerate}

\subsection{其他}
列出一些表較少見的數學表達式,用 WORD 很不容易做到。
  $$X_{n} \stackrel{d}{\longrightarrow} X$$
  
  $$\overbrace{X_{1} + \ldots + \underbrace{X_{15} + \ldots + X_{30}}}$$\\
  \begin{equation*}
    G = \left\{\begin{array}{l}
          CLASS\#1 \;\;\mbox{if} \;\; \hat{\beta}^T\bf{x} \leq 0 \\
          CLASS\#2 \;\;\mbox{if} \;\; \hat{\beta}^T\bf{x} > 0
        \end{array}\right.
  \end{equation*}\\

以 {\A equation} 或 {\A align} 排版時,數學式會自動編上號碼。文稿其他地方若要引述某數學式,可先在數學式以 $\backslash${\A label} 指令加上標籤,再使用 $\backslash${\A ref} 指令引述。如此一來若排版文稿須反覆修改,使用 $\backslash${\A label} 與$\backslash${\A ref} 指令可以「自動對焦」不會出錯。

\section{練習題}
下列八張內建數學式的圖,涵蓋一些統計領域常見的數學式細節,試著利用本章所學,細心、耐心、一步步地完成(每完成一小部分便立即編譯,才能掌握每一個看不見的錯誤)。

\begin{figure}[h]
 %   \centering
        \includegraphics[scale=0.15]{\imgdir Math_prac_5.jpg}
\end{figure}

\begin{figure}[h]
 %   \centering
        \includegraphics[scale=0.25]{\imgdir Math_prac_6.jpg}
\end{figure}

\begin{figure}[h]
 %   \centering
        \includegraphics[scale=0.8]{\imgdir Math_prac_1.png}
\end{figure}

\begin{figure}[h]
 %   \centering
        \includegraphics[scale=0.7]{\imgdir Math_prac_2.png}
\end{figure}

\begin{figure}[h]
 %   \centering
        \includegraphics[scale=0.8]{\imgdir Math_prac_3.png}
\end{figure}
\rule{\textwidth}{0.2pt}
\begin{figure}[h]
 %   \centering
        \includegraphics[scale=0.75]{\imgdir Math_prac_4.png}
\end{figure}


\begin{figure}[h]
 %   \centering
        \includegraphics[scale=0.2]{\imgdir Math_prac_8.jpg}
\end{figure}

\begin{figure}[h]
 %   \centering
        \includegraphics[scale=0.15]{\imgdir Math_prac_7.jpg}
\end{figure}
%\end{document}
